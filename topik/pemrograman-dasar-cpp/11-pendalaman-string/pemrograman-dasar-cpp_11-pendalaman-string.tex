\documentclass{beamer}
\usetheme{tokitex}

\usepackage{tikz}
\usepackage{graphics}
\usepackage{multirow}
\usepackage{tabto}

\usepackage[english,bahasa]{babel}
\newtranslation[to=bahasa]{Section}{Bagian}
\newtranslation[to=bahasa]{Subsection}{Subbagian}

\usepackage{listings, lstautogobble}
\usepackage{color}

\definecolor{dkgreen}{rgb}{0,0.6,0}
\definecolor{gray}{rgb}{0.5,0.5,0.5}
\definecolor{mauve}{rgb}{0.58,0,0.82}

\lstset{frame=tb,
  language=c++,
  aboveskip=1mm,
  belowskip=1mm,
  showstringspaces=false,
  columns=fullflexible,
  keepspaces=true,
  basicstyle={\small\ttfamily},
  numbers=none,
  numberstyle=\tiny\color{gray},
  keywordstyle=\color{blue},
  commentstyle=\color{dkgreen},
  stringstyle=\color{mauve},
  breaklines=true,
  breakatwhitespace=true,
  autogobble=true
}

\usepackage{caption}
\captionsetup[figure]{labelformat=empty}

\newcommand{\progTerm}[1]{\texttt{#1}}
\newcommand{\foreignTerm}[1]{\textit{#1}}
\newcommand{\newTerm}[1]{\alert{\textbf{#1}}}
\newcommand{\emp}[1]{\alert{#1}}
\newcommand{\statement}[1]{"#1"}


\title{Pendalaman String\newline (C++)}
\author{Tim Olimpiade Komputer Indonesia}
\date{}

\begin{document}

\begin{frame}
\titlepage
\end{frame}

\section{Pengolahan String}
\frame{\sectionpage}

\begin{frame}
\frametitle{String pada C++}
Seperti yang disinggung sebelumnya, terdapat dua macam string yang dapat digunakan di C++.
\begin{enumerate}
  \item cstring: string yang diwariskan dari bahasa C, yang merupakan leluhur bahasa C++.
  \item std string: string yang dimiliki oleh C++.
\end{enumerate}
Kita akan membahasnya satu per satu.
\end{frame}

\section{cstring}
\frame{\sectionpage}


\begin{frame}[fragile]
\frametitle{Pengenalan cstring}
\begin{itemize}
  \item Dalam bahasa C dan C++, cstring dibuat menggunakan "\foreignTerm{null terminated array of char}".
  \item Penggunaannya sesederhana membuat array \texttt{char}.
  \item Karena berasal dari bahasa C, cstring dapat berinteraksi dengan \texttt{scanf} dan \texttt{printf} secara langsung.
\end{itemize}

\begin{lstlisting}
#include <cstdio>

char s[1001];

int main() {
  scanf("%s", s);
  printf("%s\n", s);
}
\end{lstlisting}
\end{frame}

\begin{frame}
\frametitle{Representasi cstring}
\begin{itemize}
  \item Misalkan kita memasukkan "Pak Dengklek" pada program sebelumnya.
  \item Array \texttt{s} akan berisi karakter-karakter penyusun "Pak Dengklek", diakhiri dengan karakter '$\backslash$0'.
\end{itemize}
\begin{tabular}{|c|c|c|c|c|c|c|c|c|c|c|c|c|c|c|c|}
\hline 0 & 1 & 2 & 3 & 4 & 5 & 6 & 7 & 8 & 9 & 10 & 11 & 12 & 13 & 14 & ...\\ 
\hline P & a & k &  & D & e & n & g & k & l & e & k & $\backslash$ 0 & & & ...\\ 
\hline 
\end{tabular} 
\begin{itemize}
  \item Karakter '$\backslash$0' digunakan oleh C/C++ untuk menandai akhir dari string.
  \item Jadi, meskipun terdapat 1001 elemen pada \texttt{s}, C/C++ tahu string yang direpresentasikan hanya sampai elemen ke-11 saja.
\end{itemize}
\end{frame}

\begin{frame}
\frametitle{Representasi cstring}
\begin{itemize}
  \item Perhatikan bahwa \texttt{s} hanya dapat menampung paling banyak 1000 karakter.
  \item Meskipun memiliki 1001 elemen, sebuah elemen sudah dipesan untuk menempatkan karakter $\backslash$0.
  \item Apabila string yang dibaca lebih dari 1000 karakter, program tidak mengalami \foreignTerm{error}. Namun, program akan bekerja dengan tidak terprediksi.
  \item Pastikan ukuran array dibuat sebesar ukuran maksimal masukan yang mungkin.
\end{itemize}
\end{frame}

\begin{frame}[fragile]
\frametitle{Pengolahan cstring}
\begin{itemize}
  \item Seperti array pada umumnya, Anda dapat mengolah setiap elemennya secara independen.
  \item Sebagai contoh, program berikut mengubah setiap karakter dari string yang dibaca menjadi 'a'.
\end{itemize}
\begin{lstlisting}
int main() {
  scanf("%s", s);
  for (int i = 0; s[i] != '\0'; i++) {
    s[i] = 'a';
  }
  printf("%s\n", s);
}
\end{lstlisting}
\end{frame}

\begin{frame}[fragile]
\frametitle{Mencari Panjang cstring}
\begin{itemize}
  \item Kita dapat menggunakan fungsi \texttt{strlen} untuk mencari panjang suatu cstring.
  \item Fungsi ini disediakan oleh STL "cstring".
\end{itemize}
Contoh:
\begin{lstlisting}
#include <cstdio>
#include <cstring>

char s[1001];

int main() {
  scanf("%s", s);
  printf("%d\n", strlen(s));
}
\end{lstlisting}
\end{frame}

\begin{frame}[fragile]
\frametitle{Penggunaan strlen}
\begin{itemize}
  \item Fungsi strlen memiliki kompleksitas O(N), dengan N adalah panjang dari string masukan.
  \item Penyebabnya adalah strlen sebenarnya mencari di indeks keberapakah $\backslash$0 berada.
  \item Apabila hendak digunakan dalam perulangan, simpan dulu panjangnya ke dalam suatu variabel untuk menghindari pemanggilan yang berulang-ulang.
\end{itemize}
Contoh:
\begin{lstlisting}
// Buruk, O(N^2)
for (int i = 0; i < strlen(s); i++) {
  printf("%c", s[i]);
}

// Baik, O(N)
int len = strlen(s);
for (int i = 0; i < len; i++) {
  printf("%c", s[i]);
}
\end{lstlisting}
\end{frame}

\begin{frame}[fragile]
\frametitle{Membandingkan cstring}
\begin{itemize}
  \item Untuk membandingkan 2 cstring \texttt{s} dan \texttt{t} secara leksikografis, gunakan \texttt{strcmp(s,t)}.
  \item Nilai kembaliannya memiliki arti sebagai berikut:
  \begin{itemize}
    \item Negatif, artinya \texttt{s} lebih awal dari \texttt{t}.
    \item Nol, artinya \texttt{s} sama dengan \texttt{t}.
    \item Positif, artinya \texttt{s} lebih akhir dari \texttt{t}.
  \end{itemize}
  \item Kompleksitasnnya $O(N)$, dengan $N$ adalah panjang string terkecil yang dibandingkan.
\end{itemize}
Contoh penggunaan:
\begin{lstlisting}
#include <cstdio>
#include <cstring>

char s[1001];
char t[1001];

int main() {
  scanf("%s %s", s, t);
  printf("%d\n", strcmp(s, t));
}
\end{lstlisting}
\end{frame}


\begin{frame}[fragile]
\frametitle{Mengisi Array}
\begin{itemize}
  \item Untuk mengisi array \texttt{arr} dengan \texttt{x}, gunakan \texttt{memset(arr, x, sizeof(arr))}.
  \item Nilai \texttt{x} terbatas pada tipe data \texttt{char}, atau angka di antara -128 sampai 127 saja.
  \item Fungsi ini biasanya juga dimanfaatkan untuk menginisialisasi array bilangan dengan 0 atau -1.
\end{itemize}
Contoh:
\begin{lstlisting}
#include <cstdio>
#include <cstring>

char s[1001];
int arr[101];

int main() {
  memset(s, 'x', sizeof(s));
  memset(arr, -1, sizeof(arr));
  printf("%c %d\n", s[0], arr[0]);
}
\end{lstlisting}
\end{frame}

\begin{frame}[fragile]
\frametitle{Fungsi cstring Lainnya}
\begin{itemize}
  \item Terdapat beberapa fungsi cstring lain seperti \texttt{strcpy}, \texttt{strncpy}, dan \texttt{strstr}.
  \item Fungsi-fungsi tersebut tidak dibahas pada materi ini.
  \item Anda dapat mempelajarinya di dokumentasi C/C++ lebih lanjut, seperti di http://www.cplusplus.com.
\end{itemize}
\end{frame}

\section{std string}
\frame{\sectionpage}

\begin{frame}[fragile]
\frametitle{Representasi string}
\begin{itemize}
  \item String pada C++ direpresentasikan dengan struktur array dinamis.
  \item Ukurannya dapat berubah sesuai dengan ukuran string.
  \item Oleh sebab itu, Anda tidak perlu mendeklarasikan ukuran terbesar yang mungkin.
  \item Anda dapat mengakses dan mengubah nilai karakter dari suatu indeks string secara independen.
\end{itemize}
\end{frame}

\begin{frame}[fragile]
\frametitle{Mencari Panjang string}
\begin{itemize}
  \item Untuk mencari panjang dari string \texttt{s}, cukup panggil \texttt{s.length()}.
  \item Kompleksitas pemanggilannya adalah O(1).
\end{itemize}
\begin{lstlisting}
// O(N)
for (int i = 0; i < s.length(); i++) {
  printf("%c", s[i]);
}
\end{lstlisting}
\end{frame}

\begin{frame}[fragile]
\frametitle{Mencari Substring}
\begin{itemize}
  \item Untuk mencari posisi suatu substring \texttt{t} dari string \texttt{s}, gunakan \texttt{s.find(t)}.
\end{itemize}
Contoh:
\begin{lstlisting}
#include <cstdio>
#include <string>
using namespace std;

int main() {
  string s = "Pak Dengklek berternak";
  string t1 = "Dengklek";
  string t2 = "pak";
  string t3 = "klek";

  printf("%d\n", s.find(t1)); // 4
  printf("%d\n", s.find(t2)); // -1 (tak ditemukan)
  printf("%d\n", s.find(t3)); // 8
}
\end{lstlisting}
\end{frame}

\begin{frame}[fragile]
\frametitle{Mengambil Substring}
\begin{itemize}
  \item Untuk mengambil substring dari indeks \texttt{i} sebanyak \texttt{n} karakter dari string \texttt{s}, gunakan \texttt{s.substr(i, n)}.
\end{itemize}
Contoh:
\begin{lstlisting}
#include <cstdio>
#include <string>
using namespace std;

int main() {
  string s = "Pak Dengklek berternak";

  printf("%s\n", s.substr(0, 6).c_str()); // Pak De
  printf("%s\n", s.substr(2, 1).c_str()); // k
}
\end{lstlisting}
\end{frame}     

\begin{frame}[fragile]
\frametitle{Menghapus Substring}
\begin{itemize}
  \item Untuk menghapus substring dari indeks \texttt{i} sebanyak \texttt{n} karakter dari string \texttt{s}, gunakan \texttt{s.erase(i, n)}.
\end{itemize}
Contoh:
\begin{lstlisting}
#include <cstdio>
#include <string>
using namespace std;

int main() {
  string s = "Pak Dengklek berternak";
  s.erase(1, 3);

  printf("%s\n", s.c_str()); // PDengklek berternak
}
\end{lstlisting}
\end{frame}

\begin{frame}[fragile]
\frametitle{Menyisipkan String}
\begin{itemize}
  \item Untuk menyisipkan string \texttt{t} ke string \texttt{s} bermula di indeks \texttt{i}, gunakan \texttt{s.insert(i, t)}.
\end{itemize}

Contoh:
\begin{lstlisting}
#include <cstdio>
#include <string>
using namespace std;

int main() {
  string s = "Pak Dengklek berternak";
  string t = "dan Bu ";

  s.insert(4, t);
  printf("%s\n", s.c_str()); // Pak dan Bu Dengklek berternak
}
\end{lstlisting}
\end{frame}


\begin{frame}[fragile]
\frametitle{Operasi Tambahan: Penempelan String}
\begin{itemize}
  \item Pada std string, hal ini dapat dilakukan cukup dengan operasi '+', layaknya operasi numerik.
\end{itemize}
Contoh:
\begin{lstlisting}
#include <cstdio>
#include <string>
using namespace std;

int main() {
  string s = "Pak";
  string t = "Dengklek";

  string gabung = s + t;
  printf("%s\n", gabung.c_str()); // PakDengklek
}
\end{lstlisting}
\end{frame}

\begin{frame}[fragile]
\frametitle{Operasi Char}
\begin{itemize}
  \item Setiap karakter dari string memiliki tipe \texttt{char}.
  \item Kita dapat melakukan operasi penambahan atau pengurangan pada char, yang akan dioperasikan pada kode ASCII-nya.
\end{itemize}

Contoh:
\begin{lstlisting}
#include <cstdio>
#include <string>
using namespace std;

int main() {
  string s = "abc";
  s[0]++;
  s[1] += 2;
  s[2] -= 2;
  printf("%s\n", s.c_str()); // bda
}
\end{lstlisting}
\end{frame}

\begin{frame}[fragile]
\frametitle{Operasi Char (lanj.)}
\begin{itemize}
  \item Operasi ini dapat digunakan untuk mengubah karakter suatu string.
  \item Operasi yang umum adalah mengubah dari huruf kecil ke besar, dengan cara mengurangkan char dengan selisih antara ASCII 'a' dengan 'A'.
  \item Cara sebaliknya akan mengubah dari huruf besar ke huruf kecil.
  \item Hafalkan ASCII 'a' adalah 97, dan 'A' adalan 65.
\end{itemize}
\begin{lstlisting}
#include <cstdio>
#include <string>
using namespace std;

int main() {
  string s = "toki";
  for (int i = 0; i < s.size(); i++) {
    s[i] -= 'a' - 'A';
  }
  printf("%s\n", s.c_str()); // TOKI
}
\end{lstlisting}
\end{frame}

\begin{frame}
\frametitle{Selanjutnya...}
\begin{itemize}
  \item Pembelajaran kalian tentang C++ sudah cukup untuk bisa menuliskan algoritma-algoritma kompleks.
  \item Berikutnya kita akan mempelajari hal-hal yang lebih berkaitan dengan \textbf{algoritma}, bukan sekedar belajar bahasa.
\end{itemize}
\end{frame}

\end{document}
